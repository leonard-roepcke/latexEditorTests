\documentclass{article}
\usepackage[utf8]{inputenc}
\usepackage[T1]{fontenc}
%\usepackage[ngerman]{babel}
\usepackage{graphicx}
\usepackage{geometry}
\geometry{margin=25mm}

\newcommand{\AuthorName}{Leonard Röpcke\\Klasse TG-J2b}
\newcommand{\Institute}{Zepellin Gewerbeschule Konstanz}
\newcommand{\Subtitle}{Die Wissenschaft der Zahlen}
\newcommand{\MyDate}{\today}

\begin{document}
\begin{titlepage}
  \centering
  {\scshape\LARGE \Institute \par}
  \vspace{2.5cm}
  {\huge\bfseries Mathematik\par}
  \vspace{0.8cm}
  {\Large\itshape \Subtitle \par}
  \vfill
  {\Large Autor\par}
  {\Large \AuthorName \par}
  \vspace{1cm}
  {\Large Datum\par}
  {\Large \MyDate \par}
  \vfill
  % Optional: Bild einfügen (Datei hinzufügen oder Zeile auskommentieren)
  % \includegraphics[width=6cm]{example-image}
  \vspace{1cm}
  {\small }
  %unten am anfang
\end{titlepage}
\tableofcontents
\newpage

\section{Stochastik}
\subsection{Biominalverteilung}
\subsubsection{Beispiel: Uhr tragen}
In einer Schule werden \( n = 1000 \) Schülerinnen und Schüler gefragt, ob sie eine Uhr tragen.  
Die Wahrscheinlichkeit hierfür beträgt \( p = 0{,}45 \).

Gesucht ist die Wahrscheinlichkeit, dass die Anzahl der Uhrenträger:innen maximal um \( 3\sigma \) abweicht.

\[
\mu = n \cdot p = 1000 \cdot 0{,}45 = 450
\]
\[
\sigma = \sqrt{n \cdot p \cdot (1 - p)} = \sqrt{1000 \cdot 0{,}45 \cdot 0{,}55} \approx 15{,}73
\]
\[
3\sigma \approx 3 \cdot 15{,}73 = 47{,}19
\]

\noindent
\textbf{Mit der kumulierten Binomialverteilung:}\\
Mit dem Taschenrechner (gerundetes \(\sigma\) auf zwei Nachkommastellen) ergibt sich eine Wahrscheinlichkeit von etwa
\[
P(|X - \mu| \leq 3\sigma) \approx 0{,}9973 \quad \text{bzw. } 99{,}73\%.
\]

\medskip
\textbf{Mit der kumulierten Normalverteilung:}
\[
P(402 \leq X \leq 497)
= \int_{402{,}5}^{497{,}5} \varphi(x) \, dx
\approx 0{,}9987 - 0{,}0012 = 0{,}9975
\]

Damit beträgt die Wahrscheinlichkeit, dass die Anzahl der Uhrenträger:innen maximal um \( 3\sigma \) abweicht, etwa \( 99{,}75\% \).


\end{document}