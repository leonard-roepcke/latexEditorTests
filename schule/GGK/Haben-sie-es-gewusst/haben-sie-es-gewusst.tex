\documentclass{article}
\usepackage[utf8]{inputenc}
\usepackage[T1]{fontenc}
\usepackage[ngerman]{babel}
\usepackage{graphicx}
\usepackage{geometry}
\geometry{margin=25mm}

\title{Haben Sie es gewusst?}
\newcommand{\AuthorName}{Leonard Röpcke\\Klasse TG-J2b}
\newcommand{\Institute}{Zepellin Gewerbeschule Konstanz}
\newcommand{\Subtitle}{GGK}
\newcommand{\MyDate}{\today}

\begin{document}

\begin{titlepage}
  \centering
  {\scshape\LARGE \Institute \par}
  \vspace{2.5cm}
  {\huge\bfseries Haben Sie es gewusst?\par}
  \vspace{0.8cm}
  {\Large\itshape \Subtitle \par}
  \vfill
  {\Large Autor\par}
  {\Large \AuthorName \par}
  \vspace{1cm}
  {\Large Datum\par}
  {\Large \MyDate \par}
  \vfill
  % Optional: Bild einfügen (Datei hinzufügen oder Zeile auskommentieren)
  % \includegraphics[width=6cm]{example-image}
  \vspace{1cm}
  {\small }
  %unten am anfang
\end{titlepage}
\tableofcontents
\newpage

\section{Mindmap}
Quellen:
\begin{itemize}
    \item Zeitungsartikel
    \item Reden
    \item Sitzungsprotokolle
    \item Poster Hinweis: Passe Titel, Autor, Datum und andere Informationen nach Bedarf an
    \item Flugblätter
    \item Einsatzberichte
    \item Protokolle
\end{itemize}
Medien:
\begin{itemize}
    \item Tagebücher
    \item Briefe
    \item Mein Kampf
\end{itemize}
Berichte:
\begin{itemize}
    \item Zeitzeugen
    \item Feldpostbriefe
    \item nahes Umfeld
\end{itemize}
Vereinigungen:
\begin{itemize}
    \item HJ Hinweis: Passe Titel, Autor, Datum und andere Informationen nach Bedarf an
    \item NSDAP
    \item SS
    \item SA
    \item BDM
\end{itemize}

\newpage

\section{Zivilisationsbruch}
Im folgenden werde ich mich mit Folgender These auseinandersetzen: "Der Holocaust wird als "Zivilisationsbruch" bezeichnet."
Um diese These richtig zu interprettieren ist es wichtig sich damit zu beschäftigen was Zivilisationsbruch bedeutet. Folgende Definizion habe
ich erarbeitet.: Wenn ein zivilisation so stark in ihren Grundfesten erschüttert wird, dass sie sich nicht mehr von alleine erholen kann
und sich grundlegend verändern muss. Das heißt wenn wir das auf den NS beziehen würde ich aufjeden fall von einem Zivilisationsbruch sprechen,
da es die geselschaft und die Grundfesten der Gesellschaft so stark erschüttert hat, dass sie sich nicht mehr von alleine erholen konnten.
Damit meine ich diese

\end{document}