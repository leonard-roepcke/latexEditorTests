\documentclass{article}
\usepackage[utf8]{inputenc}
\usepackage[T1]{fontenc}
\usepackage[ngerman]{babel}
\usepackage{graphicx}
\usepackage{geometry}
\geometry{margin=25mm}

\title{under Titel}
\newcommand{\AuthorName}{Leonard Röpcke\\Klasse TG-J2b}
\newcommand{\Institute}{Zepellin Gewerbeschule Konstanz}
\newcommand{\Subtitle}{Thema}
\newcommand{\MyDate}{\today}

\begin{document}
\begin{titlepage}
  \centering
  {\scshape\LARGE \Institute \par}
  \vspace{2.5cm}
  {\huge\bfseries Titel\par}
  \vspace{0.8cm}
  {\Large\itshape \Subtitle \par}
  \vfill
  {\Large Autor\par}
  {\Large \AuthorName \par}
  \vspace{1cm}
  {\Large Datum\par}
  {\Large \MyDate \par}
  \vfill
  % Optional: Bild einfügen (Datei hinzufügen oder Zeile auskommentieren)
  % \includegraphics[width=6cm]{example-image}
  \vspace{1cm}
  {\small }
  %unten am anfang
\end{titlepage}
\tableofcontents
\newpage


\section{Orga}
\subsection{Themen für die KA}
\begin{itemize}
    \item Federschwingungen
    \item Holzquaderschwingungen im Wasser
    \item Schwingungen am Fadenpendel
    \item U - Rohrschwingungen
    \item Elektrische Schwingungen
    \item Elektro - Magnetische Schwingungen
\end{itemize}s
Die Formeln sind in der Formelsammlung auf Seite 22 und 32 zu finden.

\section{Schwingungen}
\subsection{Elektron - Magnetische Schwingungen}
\subsubsection{Energiebetrachtung elekttro -Magnetische Schwingung}



\end{document}