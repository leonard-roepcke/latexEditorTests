\documentclass{article}
\usepackage[utf8]{inputenc}
\usepackage[T1]{fontenc}
\usepackage[ngerman]{babel}
\usepackage{graphicx}
\usepackage{geometry}
\usepackage{geometry}
\usepackage{pgfplots}
\geometry{margin=25mm}

\title{under Titel}
\newcommand{\AuthorName}{Leonard Röpcke\\Klasse TG-J2b}
\newcommand{\Institute}{Zepellin Gewerbeschule Konstanz}
\newcommand{\Subtitle}{Thema}
\newcommand{\MyDate}{\today}

\begin{document}
\begin{titlepage}
  \centering
  {\scshape\LARGE \Institute \par}
  \vspace{2.5cm}
  {\huge\bfseries Titel\par}
  \vspace{0.8cm}
  {\Large\itshape \Subtitle \par}
  \vfill
  {\Large Autor\par}
  {\Large \AuthorName \par}
  \vspace{1cm}
  {\Large Datum\par}
  {\Large \MyDate \par}
  \vfill
  % Optional: Bild einfügen (Datei hinzufügen oder Zeile auskommentieren)
  % \includegraphics[width=6cm]{example-image}
  \vspace{1cm}
  {\small }
  %unten am anfang
\end{titlepage}
\tableofcontents
\newpage


\section{Orga}
\subsection{Themen für die KA}
\begin{itemize}
    \item Federschwingungen
    \item Holzquaderschwingungen im Wasser
    \item Schwingungen am Fadenpendel
    \item U - Rohrschwingungen
    \item Elektrische Schwingungen
    \item Elektro - Magnetische Schwingungen
\end{itemize}s
Die Formeln sind in der Formelsammlung auf Seite 22 und 32 zu finden.

\section{Schwingungen}
\section{Gedämpfte Schwingungen}
Wir haben ein Gewicht mit der Masse \( m \), welches an einer Feder mit der Federkonstante \( k \) hängt. Das gewicht hängt
in der Flüßigkeit Wasser mit einer Dichte von \( \rho \). \\
Beschleunigunskraft: $\vec{F}_b = m * \vec{a}(t)$ \\
Federkraft: $\vec{F}_F = k * \vec{s}(t)$ \\
Reibungskraft: $\vec{F}_R = b * \vec{v}(t)$ \\
Kräftebilanz: $\vec{F}_b = -\vec{F}_F -\vec{F}_R = m * \vec{a}(t) = - k * \vec{s}(t) - b * \vec{v}(t)$ \\
\\
Differentialgleichung: $m * \vec{a}(t) + k * \vec{v}(t) + k * \vec{s}(t) = 0$ \\
\begin{figure}[h]
    \centering
    \begin{tikzpicture}[scale=0.7]
        \begin{axis}[
            width=0.3\textwidth,
            height=4cm,
            xlabel={Zeit $t$},
            ylabel={$F_b$},
            title={Beschleunigungskraft}]
            \addplot[blue, samples=100, domain=0:4*pi] {cos(deg(x))*exp(-0.2*x)};
        \end{axis}
    \end{tikzpicture}
    \begin{tikzpicture}[scale=0.7]
        \begin{axis}[
            width=0.3\textwidth,
            height=4cm,
            xlabel={Zeit $t$},
            ylabel={$F_F$},
            title={Federkraft}]
            \addplot[red, samples=100, domain=0:4*pi] {-cos(deg(x))*exp(-0.2*x)};
        \end{axis}
    \end{tikzpicture}
    \begin{tikzpicture}[scale=0.7]
        \begin{axis}[
            width=0.3\textwidth,
            height=4cm,
            xlabel={Zeit $t$},
            ylabel={$F_R$},
            title={Reibungskraft}]
            \addplot[green, samples=100, domain=0:4*pi] {-sin(deg(x))*exp(-0.2*x)};
        \end{axis}
    \end{tikzpicture}
    \caption{Zeitlicher Verlauf der wirkenden Kräfte}
    \label{fig:kraefte}
\end{figure}

Es gibt verschiedene arten der Dämpfung: z.B. Konstante Dämpfung oder eine die von der Geschwindigkeit abhängig sind.
\begin{figure}[h]
    \centering
    \begin{tikzpicture}[scale=0.7]
        \begin{axis}[
            width=0.45\textwidth,
            height=6cm,
            xlabel={Zeit $t$},
            ylabel={Dämpfung},
            title={Dämpfungsarten},
            legend pos=north east]
            \addplot[blue, samples=100, domain=0:6] {1-0.2*x};
            \addplot[red, samples=100, domain=0:6] {exp(-0.5*x)};
            \legend{Lineare Dämpfung, Exponentielle Dämpfung}
        \end{axis}
    \end{tikzpicture}
    \caption{Vergleich verschiedener Dämpfungsarten}
    \label{fig:daempfung}
\end{figure}


\end{document}