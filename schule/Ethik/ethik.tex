\documentclass{article}
\usepackage[utf8]{inputenc}
\usepackage[T1]{fontenc}
%\usepackage[ngerman]{babel}
\usepackage{graphicx}
\usepackage{geometry}
\geometry{margin=25mm}


\newcommand{\AuthorName}{Leonard Röpcke\\Klasse TG-J2b}
\newcommand{\Institute}{Zepellin Gewerbeschule Konstanz}
\newcommand{\Subtitle}{Die Wissenschaft des hinterfragens}
\newcommand{\MyDate}{\today}

\begin{document}
\begin{titlepage}
  \centering
  {\scshape\LARGE \Institute \par}
  \vspace{2.5cm}
  {\huge\bfseries Ethik\par}
  \vspace{0.8cm}
  {\Large\itshape \Subtitle \par}
  \vfill
  {\Large Autor\par}
  {\Large \AuthorName \par}
  \vspace{1cm}
  {\Large Datum\par}
  {\Large \MyDate \par}
  \vfill
  % Optional: Bild einfügen (Datei hinzufügen oder Zeile auskommentieren)
  % \includegraphics[width=6cm]{example-image}
  \vspace{1cm}
  {\small }
  %unten am anfang
\end{titlepage}
\tableofcontents
\newpage

\section{Definizion von Religion}

\subsection{Geertz Definizion von Religion}
Religion ist eine Symbolsystem, das darauf ziel, 
starke umfassende und dauerhafte Stimmungen und Motivationen
 in den Menschen zu schafen, indem es Vorstellungen einer allgemeinen 
 Seinsordnung formuliert und diese Vorstellungen mir einer allgemeinen
  Seinsordnung formuliert und diese Vorstellungen mit einer solchen Aura 
  von Faktizität umgibt, dass die Stimmungen und Motivationen völlig der 
  Wiklickkeit zu entsprechen scheinen.

\subsection{Wie viele SeinsOrdnungen gibt es?}
Ich denke es gibt es unendlich viele Seinsordnungen, da jeder Mensch eine andere Wahrnehmung der Welt hat.
Jeder Mensch hat eine andere Meinung über die Welt und das Leben.

\subsection{Einfachere Definizion von Seinsordnung}
Die verbindung zwischen allem was ist. Alles was wir genau nach einer bestimmten Ordnung beschreiben können.
Man kann es so definieren, dass jeder seine eigene hat oder alle haben die selbe. Oder die Religion definiert diese für uns.

\subsection{Symbolsystem}
Ein Symbolsystem ist Ein System das Symbole benutzt um etwas zu verdeutlichen.

\subsection{das darauf ziel, starke, umfassende und dauerhafte Stimmungen und Motivationen in den Menschen zu schafen}
\begin{flushleft}
Motivationen: Sinn im Leben, leben nach dem Tod. \linebreak 
Stimmungen: positive Stimmung, Vorstellung, Vorfreude \linebreak
Kontingenz erfahrung
\end{flushleft}

\subsection{Meine Erklärung zum "diese Vorstellungen mit einer solchen Aura von Faktizität umgibt"}
Ich habe es so verstanden, dass die Religion Vorstellungen und Regeln so auf erlegt das sie als Faktische wahrheiten angesehen werden.

\subsection{die Vorstellungen einer allgemeinen Seinsordnung}
Die Vorstellung der Seinsordnung scheint ein Fakt zu sein

\subsection{Meine Erklärung zum "dass die Stimmungen und Motivationen völlig der Wiklickkeit zu entsprechen scheinen"}
Das es das Ziel von Religion ist bestimmte Stimmungen und Motivationen so glaubhat zu machen.
Das wir diese als Realität ansehen und dadurch glücklicher werden.

\end{document}