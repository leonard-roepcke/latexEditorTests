\documentclass[a4paper,12pt]{article}
\usepackage[utf8]{inputenc}
\usepackage[T1]{fontenc}
\usepackage[ngerman]{babel}
\usepackage{amsmath,amssymb}
\usepackage{tikz}
\usepackage{pgfplots}
\usetikzlibrary{decorations.pathmorphing,arrows.meta}
\usepackage{geometry}
\usepackage{minted}
\geometry{margin=25mm}

\usepackage[hidelinks]{hyperref}
%\usepackage[colorlinks=true, linkcolor=blue, urlcolor=blue, citecolor=red]{hyperref}



\begin{document}
\tableofcontents


%\autoref{sec:einleitung}

\begin{center}
  {\LARGE Kreisfläche — Grafik mit zugehöriger Formel}
\end{center}

Dies ist ein kurzes Beispieldokument. Die folgende Grafik zeigt einen Kreis mit Radius $r$; die Fläche des Kreises wird durch die bekannte Formel
\[ A = \pi r^2 \]
beschrieben. Rechts neben der Grafik steht die Formel, und in der Grafik ist der Radius eingezeichnet.

\vspace{8mm}

\begin{tikzpicture}[scale=1]
  % Kreis und Schattierung
  \def\r{2} % Radius in cm
  \fill[blue!15] (0,0) circle (\r cm);
  \draw[thick] (0,0) circle (\r cm);

  % Mittelpunkt und Punkt auf Kreis
  \fill (0,0) circle (1.5pt) node[below left] {M};
  \coordinate (P) at (\r cm,0);
  \fill (P) circle (1.5pt) node[below right] {P};

  % Radiuspfeil
  \draw[->,>=stealth,thick] (0,0) -- (P) node[midway,below] {$r$};

  % Bogen/Markierung
  \draw[decorate,decoration={snake,amplitude=0.6mm}] (0:0.9\r cm) arc (0:45:0.9\r cm);

  % Beschriftung der Fläche
  \node at (0,-2.6) {Kreis mit Radius $r$};

  % Pfeil zur Formel
  \draw[->,gray] (2.6,1.6) .. controls (4.0,1.2) and (4.0,0.2) .. (4.6,0.0);

  % Formel-Box
  \node[draw=black,rounded corners,fill=white] at (5.3,0) {
    \begin{minipage}{6.5cm}
      {\large $A=\pi r^{2}$}\\[4pt]
      Kurz: Die Fläche ergibt sich durch Integration in Polarkoordinaten:
      \[A=\int_{0}^{2\pi}\int_{0}^{r}\rho\,d\rho\,d\theta=\pi r^{2}\]
    \end{minipage}
  };
\end{tikzpicture}

\section{Test section}
%\label{sec:einleitung}
\vspace{6mm}

Kurze Herleitung (nochmals als Text): Setzt man in Polarkoordinaten $dA=\rho\,d\rho\,d\theta$, so erhält man
\[
A=\int_{0}^{2\pi}\int_{0}^{r}\rho\,d\rho\,d\theta
=\int_{0}^{2\pi}\Bigl[\tfrac{1}{2}\rho^{2}\Bigr]_{0}^{r}\,d\theta
=\int_{0}^{2\pi}\tfrac{1}{2}r^{2}\,d\theta
=\pi r^{2}.
\]

\vspace{6mm}

Weiteres: Wenn du eine andere Formel/Graphik möchtest (z.B. Parabel, Normalverteilung, elektrische Feldlinien etc.), sag mir welche Formel und ich passe die Grafik entsprechend an.


\begin{tikzpicture}
  \begin{axis}[xlabel=$x$, ylabel={$y = \sin(x)$}]
    \addplot {x^2};
  \end{axis}
\end{tikzpicture}
\section{Code}
\begin{minted}[]{python}
def hello_world():
    print("Hello, world!!!")
\end{minted}


Inline code: \mintinline{python}{print("Hello, world!")}


\end{document}
