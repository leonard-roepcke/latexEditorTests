\documentclass[a4paper,12pt]{article}

\usepackage[T1]{fontenc}
\usepackage[utf8]{inputenc}
\usepackage[ngerman]{babel}
\usepackage{xcolor}
\usepackage{listings}
\usepackage{hyperref}
\usepackage{tikz}
\usepackage{pgfplots}           

% --- Hyperlink-Setup ---
\hypersetup{
    colorlinks=true,
    linkcolor=blue,
    urlcolor=cyan,
    citecolor=green,
    pdfauthor={Dein Name},
    pdftitle={Beispieldokument mit Code und Grafiken}
}

% --- Code-Highlighting ---
\definecolor{codebg}{RGB}{245,245,244}
\definecolor{keywordcolor}{RGB}{0,0,180}
\definecolor{stringcolor}{RGB}{163,21,21}
\definecolor{commentcolor}{RGB}{0,128,0}

\lstdefinestyle{mystyle}{
  backgroundcolor=\color{codebg},
  commentstyle=\color{commentcolor}\ttfamily,
  keywordstyle=\color{keywordcolor}\bfseries,
  stringstyle=\color{stringcolor},
  basicstyle=\ttfamily\small,
  numberstyle=\tiny\color{gray},
  breaklines=true,
  frame=single,
  frameround=tttt,
  rulecolor=\color{gray!50},
  numbers=left,
  numbersep=8pt,
  showstringspaces=false,
  tabsize=4,
  captionpos=b
}

\lstset{style=mystyle}

% --- Dokument ---
\begin{document}

\tableofcontents
\newpage

\section{Einleitung}
Dies ist ein Beispiel für ein wissenschaftliches LaTeX-Dokument mit:

\begin{itemize}
    \item klickbaren Hyperlinks (\href{https://www.latex-project.org}{LaTeX-Projekt})
    \item Code-Highlighting
    \item TikZ-Grafiken
\end{itemize}

\section{Kreisfläche — Grafik mit Formel}
Hier eine anschauliche Darstellung eines Kreises:

\begin{center}
\begin{tikzpicture}[scale=1]
  % Kreis und Schattierung
  \def\r{2} % Radius
  \fill[blue!15] (0,0) circle (\r cm);
  \draw[thick] (0,0) circle (\r cm);

  % Mittelpunkt und Punkt auf Kreis
  \fill (0,0) circle (1.5pt) node[below left] {M};
  \coordinate (P) at (\r cm,0);
  \fill (P) circle (1.5pt) node[below right] {P};

  % Radiuspfeil
  \draw[->,>=stealth,thick] (0,0) -- (P) node[midway,below] {$r$};

  % Beschriftung
  \node at (0,-2.6) {Kreis mit Radius $r$};

  % Pfeil zur Formel
  \draw[->,gray] (2.6,1.6) .. controls (4.0,1.2) and (4.0,0.2) .. (4.6,0.0);

  % Formel-Box
  \node[draw=black,rounded corners,fill=white] at (5.3,0) {
    \begin{minipage}{6.5cm}
      {\large $A=\pi r^{2}$}\\[4pt]
      Kurz: Fläche durch Integration in Polarkoordinaten:
      \[A=\int_{0}^{2\pi}\int_{0}^{r}\rho\,d\rho\,d\theta=\pi r^{2}\]
    \end{minipage}
  };
\end{tikzpicture}
\end{center}

\section{Python-Code Beispiel}\label{sec:python}
Hier ein farblich hervorgehobener Python-Code:

\begin{lstlisting}[language=Python, caption={Hello World Beispiel}]
def hello_world():
    print("Hello, world!")

for i in range(3):
    hello_world()
\end{lstlisting}

Inline-Code geht auch: \lstinline{print("Hallo Welt")}

\section{Verweise im Dokument}
Siehe Abschnitt~\ref{sec:python} für ein Code-Beispiel.

\section{Diagramm-Beispiel mit PGFPlots}
\begin{tikzpicture}
\begin{axis}[xlabel=$x$, ylabel={$y=x^2$}]
    \addplot {x^2};
\end{axis}
\end{tikzpicture}

\end{document}
