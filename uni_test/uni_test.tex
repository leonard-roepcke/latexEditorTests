\documentclass{article}
\usepackage[utf8]{inputenc}
\usepackage[T1]{fontenc}
\usepackage[ngerman]{babel}
\usepackage{graphicx}
\usepackage{geometry}
\geometry{margin=25mm}

\title{under Titel}
\newcommand{\AuthorName}{Leonard Röpcke}
\newcommand{\Institute}{}
\newcommand{\header}{Informatik Studium?}
\newcommand{\Subtitle}{Universität Konstanz}
\newcommand{\MyDate}{\today}

\begin{document}
\begin{titlepage}
  \centering
  {\scshape\LARGE \Institute \par}
  \vspace{2.5cm}
  {\huge\bfseries \header \par}
  \vspace{0.8cm}
  {\Large\itshape \Subtitle \par}
  \vfill
  {\Large Autor\par}
  {\Large \AuthorName \par}
  \vspace{1cm}
  {\Large Datum\par}
  {\Large \MyDate \par}
  \vfill
  % Optional: Bild einfügen (Datei hinzufügen oder Zeile auskommentieren)
  % \includegraphics[width=6cm]{example-image}
  \vspace{1cm}
  {\small }
  %unten am anfang
\end{titlepage}
%\tableofcontents
%\newpage
\section*{Für die Uni mtischrieb test auf Latex direckt:}
\subsection{Programmieren 1}
Ich bin mit dabei in der zweiten Woche. Es wird c++ verwendet. Es werden piplines verwendet, für abgaben.
Ein typ definiert die zuläsigen Werte und die die Operatoren die ich darauf anwenden kann. Z.b(Bei int +). Denke ich.
In c++ wahr das ziehl das selbst definierte typen meist genau so schnell funktionieren wie vordefinierte.
In cpp kann man als typ auto schreiben, wenn der wert eindeutig zu einem datentyp zuzuweißen wäre. Zudem ist es bei 
generics glaube ich ganz praktisch bzw. notwendig. In c++ kann ich zb. * für eigene datentypen definieren.
\subsection{Diskrete mathematik}
Es wird gesagt das die ableitung mit d geschreiben werden kann und der Prof mein man kann jetzt mit d rechnen, wieso kann man d jetzt nicht kürzen?
\end{document}