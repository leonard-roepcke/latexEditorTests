% chapters/basics.tex
\chapter{Einleitung}

\section*{Vorwort}
Dieses Buch vermittelt die Grundlagen der digitalen Logik. Alle Inhalte basieren auf eigenständiger Recherche und praktischer Erfahrung. 
Lesende sollten stets kritisch hinterfragen und die Konzepte selbstständig nachvollziehen. 
Die digitale Logik ist ein Thema, das konsequent auf klaren, logischen Prinzipien aufbaut. 
Dieses Buch ist so aufgebaut, dass kein Vorwissen erforderlich ist, um die grundlegenden Konzepte zu verstehen. 
Praktische Übungen oder Simulationen der vorgestellten Schaltungen sind empfehlenswert, um das Verständnis zu vertiefen.

\section{Grundlagen der digitalen Logik}

\subsection{Definition der digitalen Logik}
Digitale Logik ist die Lehre von der Verarbeitung binärer Signale durch logische Operatoren. 
Die Grundlage jeder digitalen Information sind zwei Zustände: 0 und 1. 
In der Praxis werden diese Zustände häufig durch elektrische Spannungen dargestellt: ein hoher Pegel entspricht 1, ein niedriger Pegel entspricht 0. 
Digitale Logik beschreibt, wie diese Zustände miteinander verknüpft werden können, um komplexe Informationen zu verarbeiten.

\subsection{Abstraktion elektronischer Bauteile}
Um digitale Schaltungen analysieren und entwerfen zu können, abstrahieren wir elektronische Bauteile zu sogenannten logischen Gattern. 
Diese Gatter führen grundlegende logische Operationen aus und bilden die Bausteine komplexer Schaltungen. 
Die wichtigsten Gatter sind:
\begin{itemize}
    \item \textbf{UND (AND)}: Liefert 1, wenn beide Eingänge 1 sind.
    \item \textbf{ODER (OR)}: Liefert 1, wenn mindestens ein Eingang 1 ist.
    \item \textbf{NICHT (NOT)}: Kehrt den Eingangswert um.
\end{itemize}

\subsection{Wahrheitstabellen}
\begin{center}
\begin{tabular}{c c | c | c | c}
A & B & A AND B & A OR B & NOT A\\
\hline
0 & 0 & 0 & 0 & 1\\
0 & 1 & 0 & 1 & 1\\
1 & 0 & 0 & 1 & 0\\
1 & 1 & 1 & 1 & 0\\
\end{tabular}
\end{center}

\subsection{Zusammenfassung}
Die digitale Logik bildet die Grundlage aller digitalen Systeme. 
Durch die Abstraktion von elektronischen Bauteilen zu logischen Gattern lassen sich komplexe Schaltungen systematisch analysieren und entwerfen. 
Die hier vorgestellten Grundprinzipien werden in den folgenden Kapiteln weiter vertieft und durch konkrete Beispiele ergänzt.
