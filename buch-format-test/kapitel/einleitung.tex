\chapter{Einleitung}
\section{Forwort}
Guten Tag, mein Name ist Leonard Röpcke und all das wissen das ich in diesem Buch weitergebe habe ich mir selber angeeignet. 
Weshalb sie immer ein bisschen skeptisch sein sollten und ruig etwas hinterfragen sollten.
Die Digitale Logik ist meiner meinung nach einer der reinsten Themen die es so gibt, da alles logisch aufeinander aufbaut.
Ich hoffe ich werde ihnen das anhand dieses Buches näher bringen können.
Um dieses Buch zu verstehen ist es nicht notwendig vorwissen mit zubringen, aber sie sollten die einzelen Konzepte verstehen und am besten
einmal selbstständig simoliert oder sogar nachgebaut haben.
Viel spaß beim lesen und lernen wünscht ihnen Leonard Röpcke.

\section{Grundlagen der digitalen Logik}
\subsection{Was ist digitale Logik?}
Die Grundlage woraus jede Inforamtion besteht kann immer klar definiert werden. Bei einer Sprache sind es die Buchstaben, 
bei Zahlen die Ziffern.In der digitalen Logik sind es die Zustände 0 und 1. Jeder Stromquelle hatt hat eine + und eine - Seite. 
Um uns die dinge zu vereinfachen nennen wir die + Seite 1 und die - Seite 0. Die dititale Logik beschreibt jetzt wie wir diese mit Logischen
Operatoren so verknüpfen können, das wir die einst einfachsten verküpungen später in komplexe Schaltungen umwandeln können.
Die ganze digitale Logik baut darauf auf, das wir einfach schaltungen nehmen sie kombinierern und so weiter abstahieren um Komplexere vorgänge zu ermöglichen.

\subsection{Ein Transistor}

